%%%%%%%%%%%%%%%%%%%%%%%%%%%%%%%%%%%%%%%%%%%%%%%%%%%%%%%%%%%%%%%%%%%%%%%%%%%%%%%%
\documentclass[paper=a4,fontsize=11pt, hidelinks]{temp} % KOMA-article class

\usepackage[english]{babel}
\usepackage{hyperref}
\usepackage{fancyhdr}
\usepackage{fontawesome} %www.latexdraw.com/wp-content/uploads/2021/01/fontawesome5_2.pdf
% personalized date
\def\mydate{\leavevmode\hbox{\the\year-\twodigits\month-\twodigits\day}}
\def\twodigits#1{\ifnum#1<10 0\fi\the#1}
% for header/footer
\pagestyle{fancy}
\renewcommand{\headrulewidth}{0pt}
\renewcommand{\footrulewidth}{0.4pt}
\fancyhead[C]{}
\cfoot{
    In accordo con il Decreto Legge n.196 datato 30/06/2003, Autorizzo all'uso e al processamento dei miei dati personali contenuti in questo documento.
    \href{https://igor-lirussi.github.io/Curriculum-Vitae/}{Versione al \mydate. \underline{Versione aggiornata qui}} \\
    \thepage
}

%%%%%%%%%%%%%%%%%%%%%%%%%%%%%%%%%%%%%%%%%%%%%%%%%%%%%%%%%%%%%%%%%%%%%%%%%%%%%%
\begin{document}

% Upload your photo and rename it to "photo.png" or "photo.jpg"
\begin{minipage}{0.2\linewidth}
   \includegraphics[width=1\textwidth]{photo}
\end{minipage}      
\begin{minipage}{0.75\linewidth}
    \MyName{Igor Lirussi}
    \sepspace
    \noindent
    % info 
    \hfill 
    {\color{headings} Nazionalità:} Italiana | 
    {\color{headings} Nascita:} 25/12/1995 | 
    {\color{headings} Genere:} M | 
    {\color{headings} Stato Civile:} Single
    
    \hfill {\color{headings}\faEnvelope} \href{mailto:lirussi.igor@gmail.com}{lirussi.igor@gmail.com}
    | {\color{headings}\faPhone}  0039 3317055048 
    
    \hfill \href{https://www.linkedin.com/in/igor-lirussi}{{\color{headings}\faLinkedin\space LinkedIn: }igor-lirussi}| \href{https://igor-lirussi.github.io/ResearchSmall.html}{{\color{headings}\faFolder\space Portfolio: }igor-lirussi.github.io}| \href{https://github.com/igor-lirussi}{{\color{headings}\faGithub\space GitHub:} igor-lirussi}
    
    \hfill {\color{headings}\faMapMarker} via VI Maggio 24, 33030, Forgaria nel Friuli (UD), Italy
 
\end{minipage}


%%%%%%%%%%%%%%%%%%%%%%%%%%%%%%%%%%%%%%%%%%%%%%%%%%%%%%%%%%%%%%%%%%%%%%%%%%%%%%%%
\NewPart{Esperienza Lavorativa}{}
\noindent

\href{https://colors.cmpe.boun.edu.tr}{
\shortEntry{RICERCATORE - ROBOTICA COGNITIVA}
{Ott 2021 - Feb 2024}
{Istanbul (TR) Boğaziçi University: Cognitive Learning and Robotics Laboratory}
{
Sviluppo AI per robot industriali e collaborativi per svolgere azioni manipolative ed adattarsi al cambiamento dell'ambiente. Più azioni vengono unite per raggiungere obiettivi a lungo termine. \\
Sviluppo di reti neurali per riconoscimento immagini e generazione traiettorie per bracci robotici.
} {IMG/bogazici}
}
\sepspace

\href{https://colors.cmpe.boun.edu.tr}{
\shortEntry{RICERCATORE - REALTA' VIRTUALE}
{Feb 2022 - Ago 2022}
{Istanbul (TR) Boğaziçi University: BUVIAR Virtual and Augmented Reality Laboratory}
{
Sviluppo di realtà virtuale per simulazione particellare di fisica realistica per fluidi in VR.\\ Ricostruione ambienti virtuali in 3D per analisi comportamentale e videogiochi educativi.
} {IMG/bogazici}
}
\sepspace

\href{https://www.kth.se/is/rpl}{
\shortEntry{ASSISTENTE ALLA RICERCA - INTERAZIONE ROBOTICA}
{Apr 2020 - Set 2020}
{Stoccolma (SE) KTH Royal Institute of Technology: Robotic Perception and Learning Department}
{
Creazione di un software per l'interazione in linguaggio naturale con un robot umanoide.\\ Sviluppo chatbot per robot con riconoscimento parlato, generazione risposta e sintesi vocale.
} {IMG/kth}
}
\sepspace

\href{https://welcome.isr.tecnico.ulisboa.pt/}{
\shortEntry{ASSISTENTE ALLA RICERCA - COMPUTER VISION}
{Set 2018 - Set 2019 }
{Lisbon (PT) IST Instituto Superior Técnico: I.S.R. Institute for Systems and Robotics - VisLab}
{
Navigazione di robot mobili nell'ambiente con sistemi di visione e localizzazione bluetooth per movimento e interazione con persone al suo interno.
} {IMG/ist}
}
%%%%%%%%%%%%%%%%%%%%%%%%%%%%%%%%%%%%%%%%%%%%%%%%%%%%%%%%%%%%%%%%%%%%%%%%%%%%%%%%
\NewPart{EDUCAZIONE}{}
\noindent

\href{https://corsi.unibo.it/2cycle/ComputerScienceEngineering}{
\shortEntry{Magistrale INGEGNERIA E SCIENZE INFORMATICHE}
{Set 2019 - oggi}
{Università di Bologna}
{Distributed Systems, Machine Learning,    Languages, Compilers and Computational Models,    Information Systems,     Concurrent and Distributed Programming,    Programming and Development paradigms,    Web Services and Applications,     Smart City and Mobile Technologies,     Agile, Continuous Integration and Delivery,  } 
{IMG/unibo}
}
\sepspace

\href{https://dsv.su.se/en/}{
\shortEntry{ERASMUS Magistrale}
{Gen 2020 - Lug 2020}
{Università di Stoccolma}
{Decision Making and Business Intelligence, Network Security, Cyber Forensics}
{IMG/stockholmuni}
}
\sepspace

\href{https://ciencias.ulisboa.pt/en}{
\shortEntry{ERASMUS Triennale}
{Set 2018 - Set 2019}
{Università di Lisbona}
{Artificial Intelligence, Operational Research} 
{IMG/ulisboa}
}
\sepspace

\href{https://corsi.unibo.it/1cycle/ComputerScienceEngineering}{
\shortEntry{Triennale INGEGNERIA E SCIENZE INFORMATICHE}
{Set 2014 - Set 2019}
{Università di Bologna}
{Software Engineering, Embedded Systems and IoT, Automatic Controls, Mobile Application Programming, Operating Systems, Object-Oriented Programming, Network Programming, Telecommunications Networks, Law for Information Technology, Fundamentals of Image Processing, Databases, Algorithms and Data Structures, Computer Architecture, C Programming}
{IMG/unibo}
}

%%%%%%%%%%%%%%%%%%%%%%%%%%%%%%%%%%%%%%%%%%%%%%%%%%%%%%%%%%%%%%%%%%%%%%%%%%%%%%%%
\NewPart{ Abilità Tecniche \& Software }{}
\begin{minipage}[t]{0.67\textwidth} 
    \begin{itemize}
    \item Programmazione: imperativa/OO/funzionale/logica/concorrente.
    \item Linguaggi di programmazione: Java, Scala, Python, Bash scripting.
    \item Conoscenza avanzata di algoritmi di Machine Leargning, processamento immagini e visione artificiale.
    \item Padronanza di software per l'automazione, microcontrollori, sistemi embedded e funzionamento di sensori e attuatori meccanici.
    \end{itemize}
\end{minipage}
%
\begin{minipage}[t]{0.32\textwidth} 
\begin{tabular}[t]{ l l }
\software{IMG/software/git}  & Git/GitHub Actions-Projects\\
\software{IMG/software/opencv}  & OpenCV\\
\software{IMG/software/unity} & Unity \\
\software{IMG/software/ros}  & ROS\\
\software{IMG/software/pytorch}  & PyTorch \\
\software{IMG/software/latex}  & Latex Writing \\
\software{IMG/software/office} & Office Suite
\end{tabular}
\end{minipage}

%%%%%%%%%%%%%%%%%%%%%%%%%%%%%%%%%%%%%%%%%%%%%%%%%%%%%%%%%%%%%%%%%%%%%%%%%%%%%%%%
\NewPart{Lingue e Interessi}{}
\hspace{3mm}
\begin{minipage}[t]{0.33\textwidth} 
%Lingue (ISO 639-1)
\begin{tabular}[t]{ l c l }
\flag{IMG/flag/it} & IT & Madrelingua \\
\flag{IMG/flag/gb} & EN & \href{https://github.com/igor-lirussi/Curriculum-Vitae/raw/main/Certificates/IELTS_LIRUSSI.pdf}{Competenza completa}\\
\flag{IMG/flag/pt} & PT & \href{https://github.com/igor-lirussi/Curriculum-Vitae/raw/main/Certificates/cert_PT_LIRUSSI.pdf}{Livello conversazione}\\
\flag{IMG/flag/sv} & SV & \href{https://github.com/igor-lirussi/Curriculum-Vitae/raw/main/Certificates/cert_SE_LIRUSSI.pdf}{Livello Base}\\
\flag{IMG/flag/tr} & TR & \href{https://github.com/igor-lirussi/Curriculum-Vitae/raw/main/Certificates/cert_TR_LIRUSSI.pdf}{Livello Base}\\
%\flag{IMG/flag/tr} & TR & Basic level\\
%\flag{IMG/flag/de} & DE & A2 & \href{https://github.com/igor-lirussi/Curriculum-Vitae/raw/main/Certificates/cert_DE_LIRUSSI.pdf}{Basic level}\\
\end{tabular}
\end{minipage}
%
\begin{minipage}[t]{0.64\textwidth} 
\begin{tabular}[t]{l l}
\href{https://site.unibo.it/startupdayunibo/en/}{Vincitore "Le nuove 30 idee emergenti 2021, StartUp Day - Unibo"}\\
Corso di primo soccorso per aziende teorico e pratico.\\
Appassionato di fai-da-te e cucina.\\
Sport e videomaking.\\
Donatore di sangue e volontario al Banco Alimentare.\\
\end{tabular}
\end{minipage}


%%% References

\end{document}
